%%%%%%%%%%%%  Generated using docx2latex.com  %%%%%%%%%%%%%%

%%%%%%%%%%%%  v2.0.0-beta  %%%%%%%%%%%%%%

\documentclass[12pt]{article}
\usepackage{amsmath}
\usepackage{latexsym}
\usepackage{amsfonts}
\usepackage[normalem]{ulem}
\usepackage{soul}
\usepackage{array}
\usepackage{amssymb}
\usepackage{extarrows}
\usepackage{graphicx}
\usepackage[backend=biber,
style=numeric,
sorting=none,
isbn=false,
doi=false,
url=false,
]{biblatex}\addbibresource{bibliography.bib}

\usepackage{subfig}
\usepackage{wrapfig}
\usepackage{wasysym}
\usepackage{enumitem}
\usepackage{adjustbox}
\usepackage{ragged2e}
\usepackage[svgnames,table]{xcolor}
\usepackage{tikz}
\usepackage{longtable}
\usepackage{changepage}
\usepackage{setspace}
\usepackage{hhline}
\usepackage{multicol}
\usepackage{tabto}
\usepackage{float}
\usepackage{multirow}
\usepackage{makecell}
\usepackage{fancyhdr}
\usepackage[toc,page]{appendix}
\usepackage[hidelinks]{hyperref}
\usetikzlibrary{shapes.symbols,shapes.geometric,shadows,arrows.meta}
\tikzset{>={Latex[width=1.5mm,length=2mm]}}
\usepackage{flowchart}\usepackage[paperheight=11.69in,paperwidth=8.27in,left=0.98in,right=0.98in,top=0.98in,bottom=0.98in,headheight=1in]{geometry}
\usepackage[utf8]{inputenc}
\usepackage[T1]{fontenc}
\TabPositions{0.49in,0.98in,1.47in,1.96in,2.45in,2.94in,3.43in,3.92in,4.41in,4.9in,5.39in,5.88in,}

\urlstyle{same}

\renewcommand{\_}{\kern-1.5pt\textunderscore\kern-1.5pt}

 %%%%%%%%%%%%  Set Depths for Sections  %%%%%%%%%%%%%%

% 1) Section
% 1.1) SubSection
% 1.1.1) SubSubSection
% 1.1.1.1) Paragraph
% 1.1.1.1.1) Subparagraph


\setcounter{tocdepth}{5}
\setcounter{secnumdepth}{5}


 %%%%%%%%%%%%  Set Depths for Nested Lists created by \begin{enumerate}  %%%%%%%%%%%%%%


\setlistdepth{9}
\renewlist{enumerate}{enumerate}{9}
		\setlist[enumerate,1]{label=\arabic*)}
		\setlist[enumerate,2]{label=\alph*)}
		\setlist[enumerate,3]{label=(\roman*)}
		\setlist[enumerate,4]{label=(\arabic*)}
		\setlist[enumerate,5]{label=(\Alph*)}
		\setlist[enumerate,6]{label=(\Roman*)}
		\setlist[enumerate,7]{label=\arabic*}
		\setlist[enumerate,8]{label=\alph*}
		\setlist[enumerate,9]{label=\roman*}

\renewlist{itemize}{itemize}{9}
		\setlist[itemize]{label=$\cdot$}
		\setlist[itemize,1]{label=\textbullet}
		\setlist[itemize,2]{label=$\circ$}
		\setlist[itemize,3]{label=$\ast$}
		\setlist[itemize,4]{label=$\dagger$}
		\setlist[itemize,5]{label=$\triangleright$}
		\setlist[itemize,6]{label=$\bigstar$}
		\setlist[itemize,7]{label=$\blacklozenge$}
		\setlist[itemize,8]{label=$\prime$}

\setlength{\topsep}{0pt}\setlength{\parskip}{8.04pt}
\setlength{\parindent}{0pt}

 %%%%%%%%%%%%  This sets linespacing (verticle gap between Lines) Default=1 %%%%%%%%%%%%%%


\renewcommand{\arraystretch}{1.3}


%%%%%%%%%%%%%%%%%%%% Document code starts here %%%%%%%%%%%%%%%%%%%%



\begin{document}
\section*{Specyfikacja funkcjonalna programu trawniczek}
\addcontentsline{toc}{section}{Specyfikacja funkcjonalna programu trawnicz}
\begin{Center}
Dragun Maciej, Piłka Hubert, Smoliński Mateusz
\end{Center}\par

\begin{Center}
16.04.2020
\end{Center}\par


\vspace{\baselineskip}
\subsection*{Cel projektu}
\addcontentsline{toc}{subsection}{Cel projektu}
Program ma na celu wyznaczenie rozmieszczenia podlewaczek na prostokątnej działce, tak aby był on jak najrównomierniej podlany. Program wykonywany jest w linii poleceń z ścieżką do pliku tekstowego będącego rozkładem działki w postaci opisanej poniżej i ewentualnymi innymi argumentami, opisanymi poniżej. Jako wynik program zwraca plik tekstowy zawierający rozkład podlewaczek i bitmapę wizualizującą stan nawodnienia przy takim rozkładzie.\par

\subsection*{Działanie programu}
\addcontentsline{toc}{subsection}{Działanie programu}
Program analizuje dwuwymiarowy, prostokątny obszar nazywany inaczej działką. Użytkownik przekazuje programowi rozkład działki w postaci pliku tekstowego i opcjonalnie liczbę okresów pełnej polewaczki i jej promień w momencie jego uruchomienia. Działkę można podzielić na rozłączne, leżące jeden za drugim poziomo i pionowo kwadraty 100x100 pikseli. Znaki w pliku wejściowym oznaczają czy wszystkie piksele w danym kwadracie są trawnikiem – częścią działki, którą należy podlać czy przeszkodą – częścią której podlać się nie da. Minimalne rozmiary działki ro 100x100 pikseli, a maksymalne 8000x4000 pikseli (80 kolumn, 40 rzędów znaków). Program następnie przystępuje do rozmieszczania podlewaczek. Są to wycinki koła o ustalonych promieniach i kształtach, które zwiększają wartość ‘nawodnienia’ każdego piksela będącego trawnikiem w fragmencie działki, który podlewaczka pokrywa. Program będzie dążył do rozmieszczenia polewaczek w taki sposób, żeby każdy wartość ‘nawodnienia’ każdego piksela trawnika była zbliżona do średniej nawodnienia całego trawnika i większa od zera. Podlewaczki umieszczane są w pikselach działki.\par

Dostępne są cztery rodzaje podlewczek:\par

\begin{itemize}
	\item „Pełna$"$  podlewaczka – koło 360$ ^{\circ} $ , promień 200 pikseli lub wartość podana (opis w dane wejściowe). Moc 1.\par

	\item Podlewaczka 270$ ^{\circ} $  - wycinek kołowy o kącie 270$ ^{\circ} $ , promień 1,5x promień „pełnej$"$  podlewaczki (300). Moc 2.\par

	\item Podlewaczka 180$ ^{\circ} $  - wycinek kołowy o kącie 180$ ^{\circ} $ , promień 2x promień „pełnej$"$  podlewaczki (400). Moc XXXXXXXX\par

	\item Podlewaczka 90$ ^{\circ} $  - wycinek kołowy o kącie 90$ ^{\circ} $ , promień 2,5x promień „pełnej$"$  podlewaczki (500). Moc XXXXXX
\end{itemize}\par

Umieszczenie podlewaczki przez program zwiększa wartość wszystkich pikseli w obejmowanym obszarze będących trawnikiem o ilość cykli podlewania pełnej polewaczki razy moc podlewaczki.\par

\subsection*{Dane wejściowe}
\addcontentsline{toc}{subsection}{Dane wejściowe}
Do działania program wymaga informacji o kształcie trawnika, ilości wykonanych cyki i promieniu pełnej podlewaczki. Są one przekazane w następującej formie:\par

\begin{itemize}
	\item Ścieżka do pliku tekstowego lub jego nazwa, jeżeli znajduje się w tym samym folderze, który przedstawia układ działki. Plik zawiera znaki ‘\textit{x}’ oraz ‘\textit{-}’\textit{ }umieszczone w co najwyżej 40 równej długości wierszach. Maksymalna długość wierszy to 80 znaków ‘x’ i ‘-‘ plus znak końca linii. Plik powinien zawierać co najmniej jeden znak ‘x’ lub ‘-‘. Każdy ze znaków interpretowany jest jako fragment działki 100x100 pikseli, który dla ‘x’ oznacza przeszkodę, a dla ‘-‘ niepodlany trawnik.\par

	\item Liczba całkowita oznaczająca ilość cykli „pełnej$"$  podlewaczki. Pozostałe podlewaczki podlewają zależnie od tej wartości. Jeżeli argument nie zostanie podany będzie wynosił domyślnie 100.\par

	\item Liczba całkowita oznaczająca promień „pełnej$"$  podlewaczki. Pozostałe podlewaczki mają promień zależny od tej wartości. Jeżeli argument nie zostanie podany będzie wynosił domyślnie 200 pikseli.
\end{itemize}\par


\vspace{\baselineskip}
\subsection*{Wynik działania programu}
\addcontentsline{toc}{subsection}{Wynik działania programu}
Jeżeli działanie programu zostało zakończone poprawnie powinien on zwrócić dwa pliki:\par

\begin{itemize}
	\item Plik tekstowy w którym w oddzielnych liniach wypisane jest rodzaj i współrzędne kolejnych podlewaczek. XXXXXXXXX\par

	\item Plik bitmapy w którym w graficzny sposób został przedstawiony stan końcowy trawnika. XXXXXXX
\end{itemize}\par


\vspace{\baselineskip}
\subsection*{Błędy}
\addcontentsline{toc}{subsection}{Błędy}
\begin{itemize}
	\item „No imput file$"$  – ścieżka / nazwa pliku nie została podana jako argument w linii poleceń. Program kończy działanie.\par

	\item "Couldn't open input file."\ – program nie mógł  otworzyć pliku o podanej nazwie. Program kończy działanie.\par

	\item $``$The Job could not be done$"$  – program nie był w stanie wyznaczyć pozycji podlewaczek. Program kończy działanie.\par

	\item $``$Could not open output files$"$  – co najmniej jeden z plików wyjścia nie mógł zostać otwarty. Program kończy działanie.\par

	\item „Could not create bitmap$"$  – Bitmapa nie mogła zostać stworzona. Program kończy działanie.\par

	\item „Could not allocate memory$"$  – program nie był w stanie przeznaczyć pamięci. Program kończy działanie.\par

	\item „Line n is to long/short$"$  – w n-tym wierszu pliku wejściowego znajduje się inna liczba znaków niż w pierwszym. Należy podać plik zgodny z opisem w dokumentacji. Program kończy działanie.\par

	\item „File has too many lines$"$  – plik wejściowy ma za dużo wierszy. Należy podać plik zgodny z opisem w dokumentacji. Program kńczy działanie.\par

	\item „Unrecognized character in $\%$ d. line.$"$  – w pliku wejściowym znajduje się niepożądany znak (inny niż ‘x’, ‘-‘ lub znak końca linii). Należy podać plik zgodny z opisem w dokumentacji. Program kończy działanie.\par

	\item „File is empty$"$  – program nie znalazł żadnego znaku w podanym pliku wejściowym. W pliku wejściowym musi znaleźć się co najmniej jeden znak ‘x’ lub ‘-‘. Program kończy działanie.
\end{itemize}\par


\printbibliography
\end{document}